\documentclass[10pt]{article}\usepackage{graphicx, color}
%% maxwidth is the original width if it is less than linewidth
%% otherwise use linewidth (to make sure the graphics do not exceed the margin)
\makeatletter
\def\maxwidth{ %
  \ifdim\Gin@nat@width>\linewidth
    \linewidth
  \else
    \Gin@nat@width
  \fi
}
\makeatother

\IfFileExists{upquote.sty}{\usepackage{upquote}}{}
\definecolor{fgcolor}{rgb}{0.2, 0.2, 0.2}
\newcommand{\hlnumber}[1]{\textcolor[rgb]{0,0,0}{#1}}%
\newcommand{\hlfunctioncall}[1]{\textcolor[rgb]{0.501960784313725,0,0.329411764705882}{\textbf{#1}}}%
\newcommand{\hlstring}[1]{\textcolor[rgb]{0.6,0.6,1}{#1}}%
\newcommand{\hlkeyword}[1]{\textcolor[rgb]{0,0,0}{\textbf{#1}}}%
\newcommand{\hlargument}[1]{\textcolor[rgb]{0.690196078431373,0.250980392156863,0.0196078431372549}{#1}}%
\newcommand{\hlcomment}[1]{\textcolor[rgb]{0.180392156862745,0.6,0.341176470588235}{#1}}%
\newcommand{\hlroxygencomment}[1]{\textcolor[rgb]{0.43921568627451,0.47843137254902,0.701960784313725}{#1}}%
\newcommand{\hlformalargs}[1]{\textcolor[rgb]{0.690196078431373,0.250980392156863,0.0196078431372549}{#1}}%
\newcommand{\hleqformalargs}[1]{\textcolor[rgb]{0.690196078431373,0.250980392156863,0.0196078431372549}{#1}}%
\newcommand{\hlassignement}[1]{\textcolor[rgb]{0,0,0}{\textbf{#1}}}%
\newcommand{\hlpackage}[1]{\textcolor[rgb]{0.588235294117647,0.709803921568627,0.145098039215686}{#1}}%
\newcommand{\hlslot}[1]{\textit{#1}}%
\newcommand{\hlsymbol}[1]{\textcolor[rgb]{0,0,0}{#1}}%
\newcommand{\hlprompt}[1]{\textcolor[rgb]{0.2,0.2,0.2}{#1}}%

\usepackage{framed}
\makeatletter
\newenvironment{kframe}{%
 \def\at@end@of@kframe{}%
 \ifinner\ifhmode%
  \def\at@end@of@kframe{\end{minipage}}%
  \begin{minipage}{\columnwidth}%
 \fi\fi%
 \def\FrameCommand##1{\hskip\@totalleftmargin \hskip-\fboxsep
 \colorbox{shadecolor}{##1}\hskip-\fboxsep
     % There is no \\@totalrightmargin, so:
     \hskip-\linewidth \hskip-\@totalleftmargin \hskip\columnwidth}%
 \MakeFramed {\advance\hsize-\width
   \@totalleftmargin\z@ \linewidth\hsize
   \@setminipage}}%
 {\par\unskip\endMakeFramed%
 \at@end@of@kframe}
\makeatother

\definecolor{shadecolor}{rgb}{.97, .97, .97}
\definecolor{messagecolor}{rgb}{0, 0, 0}
\definecolor{warningcolor}{rgb}{1, 0, 1}
\definecolor{errorcolor}{rgb}{1, 0, 0}
\newenvironment{knitrout}{}{} % an empty environment to be redefined in TeX

\usepackage{alltt}
\usepackage{fullpage}
%\usepackage[pdftex]{graphicx}
%\usepackage{graphicx}

\usepackage{henrian-basic}
\usepackage{henrian-homework}

\newcommand{\rinline}[1]{SOMETHING WRONG WITH KNITR}



\usepackage{amsmath}
\usepackage{amssymb}
\usepackage{subfig}
\usepackage{hyperref}
\usepackage{enumerate}
\usepackage{cancel}
\usepackage{float}
\usepackage{booktabs}
\usepackage[bottom]{footmisc}

\usepackage{beramono}
\usepackage{listings}
\lstset{basicstyle=\ttfamily}

\usepackage{natbib}
\bibliographystyle{plainnat}
\bibpunct{(}{)}{,}{a}{,}{,}

\makeHeaders{NLP: Homework 2}

\begin{document}

\section{HMMs and CRFs}

A Hidden Markov Model (HMM) can be used a generative model of a system of linear symbols, like language.







\begin{table}[H]
  \begin{tabular}{lllllr}
    Model & Corpus & Training accuracy & Testing accuracy & OOV-only & Time \\
    \midrule
    HMM   & Atis & 0.888 & 0.865 &
      0.180 & 2.987 \\
    CRF   & Atis & 0.999 & 0.928 &
      0.273 & 84.971 \\
    CRF+ & Atis & 0.999 & 0.940 &
      0.440 & 84.570 \\
    \midrule
    HMM   & WSJ & 0.862 & 0.785 &
      0.379 & 51.304 \\
    CRF   & WSJ & 0.951 & 0.752 &
      0.431 & 1604.749 \\
    CRF+ & WSJ & 0.992 & 0.862 &
      0.626 & 5608.262 \\
  \end{tabular}
  \caption{`CRF+' refers to the CRF with additional word-property features; `OOV-only' refers to accuracy on out-of-vocabulary items in the held-out test data.}
\end{table}




\begin{lstlisting}
  git clone https://github.com/chbrown/nlp.git
\end{lstlisting}

\end{document}

cross:
  corpus:     wsj / atis
  model:      crf / hmm
  features:   with ortho / normal / only ortho

measurements:
  run time
  variables
  oov:        all / only oov
  train/test: train / both

Report

Your report should contain a problem statement, the approach that you took to solve the problem, a brief discussion of the experiments you ran, including a nicely formatted table(s) of results, with training and test accuracy (both overall and just for OOV items), number of OOV items as a percentage of total test tokens, and run time for each approach for each corpus. Also include a discussion of your results that answers at least the following questions, where possible, explaining the results in terms of underlying algorithmic properties.
1. How does the overall test accuracy of CRF and HMM differ (when using only tokens) and why?
2. How does the test accuracy for OOV items for CRF and HMM differ (when using only tokens) and why?
3. How does the training accuracy of HMM and CRF differ and why?
4. How does the run time of HMM and CRF differ and why?
5. How does adding orthographic features affect the accuracy (both overall and OOV) and runtime of the CRF and why?
Which features helped the most? (i.e. try only including some feature types and not others)
If you are ambitious and have time, you might try adding additional types of features of your own design and discuss your results.

